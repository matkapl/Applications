\documentclass[11pt,letterpaper]{article}
\usepackage{graphicx}
%\usepackage{fullpage}
\pagenumbering{gobble}
\usepackage{geometry}
\usepackage{helvetica}
\geometry{
  top=.8in,            % <-- you want to adjust this
  inner=.8in,
  outer=.8in,
  bottom=.3in,
  headheight=2ex,       % <-- and this
  headsep=2ex,          % <-- and this
}
\begin{document}

\noindent Matias Kaplan\\
\noindent matiaskaplan92@gmail.com\\
\medskip


\begin{flushleft}

\noindent	I am passionate about harnessing biological discoveries to help solve current and future medical challenges. After graduating in December 2014 with a B.S. in Bioengineering and a minor in physics from the University of Florida, I plan on obtaining a PhD. I aim to one day run my own bioengineering lab, where I can work on genomic engineering tools and mentor the next generation of researchers.   \\ 
\bigskip

\noindent	I became interested in viruses in high school after reading The Giving Plague by David Brin a short sci-fi story about a fictional virus that induced altruistic behavioral effects due to its mode of transmission. \\
\bigskip

\noindent During my first week of freshmen year, I pursued my interest in viruses by joining Dr. Mavis Agbandje-McKenna’s lab in the Department of Biochemistry and Molecular Biology. We worked on projects using structural biology to improve Adeno-Associated Virus (AAV) as a gene therapy vector.  \\
\bigskip

\noindent Early in my undergraduate career, I also became involved with the HHMI-Science for Life Program at UF, after receiving an Intramural Award for my researcher in Dr. McKenna’s lab. I began to develop a strong appreciation for scientific community and collaboration, as I met students and professors from departments across campus. During my sophomore year, I worked to create the first research club at UF, The Science for Life Gator Student Research Club.  \\
\bigskip

\noindent 		This past summer, I entered a larger scientific community when I traveled to the University of California, Berkeley as an HHMI-EXROP scholar to work in Dr. Jennifer Doudna’s on the first structural studies of Cas9. My time in Dr. Doudna’s lab made me realize how powerful it was to research tools necessary to do genomic engineering work. The effect that Cas9 has had on the field made me aware of the need for bioengineering tools. \\
\bigskip

\noindent My research interests have evolved during the past three years. Initially, I became interested in studying the basic biology of virus-human interactions. Now, my main research interest is in developing new bioengineering tools to help make the research process more efficient and accessible for scientists. \\
\bigskip

\noindent It was not until our society gained access to more comprehensive and simplified programming tools that computers became a transformative power; analogously, a deeper and more accessible understanding of bioengineering tools has been and will continue to be a catalyst for humanity’s advancement. \\
\bigskip

\noindent I am eager to learn from the researchers at ASGCT whom I admire; I deeply value being a part of a scientific community, and I am excited to attend this conference as a significant step towards pursuing a career in genomic engineering. I look forward to working with the next generation of creative and passionate researchers who love their work, from the elegance of a virus to the power of the tools that help make change. \\
\bigskip

\end{flushleft}
\smallskip

\iffalse
\noindent Sincerely,

\begin{figure}[h!]

\includegraphics[scale=0.05]{matias.png}
\end{figure}

\noindent Matias Kaplan
\fi

\end{document}
